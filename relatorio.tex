\documentclass{article}
\usepackage[utf8]{inputenc}
\usepackage[portuguese]{babel}
\usepackage{amsmath}
\usepackage{geometry}
\geometry{a4paper, margin=2cm}

\title{Relatório Técnico: Implementação 
de Matrizes em C++}
\author{Equipe}
\date{\today}

\begin{document}

\maketitle

\section{Introdução}
Este relatório descreve a implementação de uma biblioteca para manipulação de matrizes em C++, incluindo:
\begin{itemize}
    \item Estruturas de dados utilizadas
    \item Divisão de módulos
    \item Descrição das rotinas
    \item Análise de complexidade (tempo e espaço)
\end{itemize}

\section{Estruturas de Dados Utilizadas}

\subsection{Classe \texttt{Matriz}}
Representa uma matriz genérica de dimensões $m \times n$.

\textbf{Atributos:}
\begin{itemize}
    \item \texttt{linhas} (int): Número de linhas
    \item \texttt{colunas} (int): Número de colunas
    \item \texttt{dados} (\texttt{float**}): Matriz dinâmica alocada como um array de ponteiros
\end{itemize}

\subsection{Classe \texttt{MatrizQuadrada} (Herda de \texttt{Matriz})}
Representa matrizes quadradas ($n \times n$), com operações específicas:

\textbf{Métodos adicionais:}
\begin{itemize}
    \item \texttt{traco()}: Calcula o traço da matriz
    \item \texttt{determinante()}: Calcula o determinante (implementado apenas para $2\times2$)
\end{itemize}

\section{Divisão de Módulos}
O código está organizado em:
\begin{enumerate}
    \item Definição da Classe \texttt{Matriz} (operações básicas e alocação dinâmica)
    \item Classe \texttt{MatrizQuadrada} (extensão para matrizes quadradas)
    \item Operações Matemáticas (sobrecarga de operadores)
    \item Interface do Usuário (menu interativo no \texttt{main()})
\end{enumerate}

\section{Descrição das Rotinas Principais}

\subsection{Operações Básicas (\texttt{Matriz})}
\begin{tabular}{|l|p{10cm}|}
\hline
\textbf{Método/Função} & \textbf{Descrição} \\
\hline
\texttt{Matriz(int l, int c)} & Construtor que aloca memória para uma matriz $l \times c$ \\
\hline
\texttt{\~Matriz()} & Destrutor que libera a memória alocada \\
\hline
\texttt{preencher(string)} & Preenche a matriz a partir de uma string no formato \texttt{[[a,b],[c,d]]} \\
\hline
\texttt{alterar(i, j, valor)} & Modifica o valor na posição $(i, j)$ \\
\hline
\texttt{imprimir()} & Exibe a matriz no console \\
\hline
\end{tabular}

\subsection{Operações Matemáticas}
\begin{tabular}{|l|p{10cm}|}
\hline
\textbf{Operação} & \textbf{Descrição} \\
\hline
\texttt{A + B} (\texttt{operator+}) & Soma duas matrizes (retorna uma nova matriz ou erro se dimensões incompatíveis) \\
\hline
\texttt{A - B} (\texttt{operator-}) & Subtrai duas matrizes \\
\hline
\texttt{A * escalar} & Multiplica a matriz por um escalar \\
\hline
\texttt{A * B} & Multiplicação de matrizes (retorna \texttt{nullptr} se dimensões incompatíveis) \\
\hline
\texttt{transposicao(A)} & Retorna a matriz transposta \\
\hline
\end{tabular}

\subsection{Métodos Específicos (\texttt{MatrizQuadrada})}
\begin{tabular}{|l|p{10cm}|}
\hline
\textbf{Método} & \textbf{Descrição} \\
\hline
\texttt{traco()} & Soma os elementos da diagonal principal \\
\hline
\texttt{determinante()} & Calcula o determinante (apenas para matrizes $2\times2$) \\
\hline
\end{tabular}

\section{Análise de Complexidade}

\subsection{Complexidade de Tempo}
\begin{tabular}{|l|l|p{8cm}|}
\hline
\textbf{Operação} & \textbf{Complexidade} & \textbf{Explicação} \\
\hline
\texttt{Matriz::Matriz(l, c)} & $O(l \times c)$ & Alocação de memória para $l \times c$ elementos \\
\hline
\texttt{Matriz::preencher(str)} & $O(l \times c)$ & Processa cada elemento da string e armazena na matriz \\
\hline
\texttt{A + B / A - B} & $O(n^2)$ & Percorre todos os elementos das duas matrizes \\
\hline
\texttt{A * escalar} & $O(n^2)$ & Percorre todos os elementos da matriz \\
\hline
\texttt{A * B} & $O(n^3)$ & Triplo loop para multiplicação de matrizes \\
\hline
\texttt{transposicao(A)} & $O(n^2)$ & Percorre todos os elementos e os reposiciona \\
\hline
\texttt{traco()} & $O(n)$ & Percorre apenas a diagonal principal \\
\hline
\texttt{determinante()} & $O(1)$ & Cálculo direto para matriz $2\times2$ \\
\hline
\end{tabular}

\subsection{Complexidade de Espaço}
\begin{tabular}{|l|l|p{8cm}|}
\hline
\textbf{Operação} & \textbf{Complexidade} & \textbf{Explicação} \\
\hline
\texttt{Matriz(l, c)} & $O(l \times c)$ & Armazena uma matriz $l \times c$ \\
\hline
\texttt{A + B / A - B} & $O(n^2)$ & Gera uma nova matriz de mesmo tamanho \\
\hline
\texttt{A * B} & $O(n^2)$ & Retorna uma matriz de tamanho $\text{lin\_A} \times \text{col\_B}$ \\
\hline
\texttt{transposicao(A)} & $O(n^2)$ & Gera uma nova matriz transposta \\
\hline
\end{tabular}


\end{document}